\documentclass{article}

\usepackage{luatexbase}
\usepackage{fontspec}
\ExplSyntaxOn
\keys_define:nn {fontspec-renderer}
 {
  Renderer / harf .code:n =
   {
      \tl_set:Nx \l_fontspec_mode_tl {harf}
      \tl_gset:Nx \g__fontspec_single_feat_tl { mode=harf }
   }
 }

\ExplSyntaxOff
\directlua{require("harf-luaotfload.lua")}


\setmainfont{Amiri}[Script=Arabic,Renderer=harf]
\begin{document}
\def\arabictext{%
هذا كتاب صغير في بحث جديد، تنبّهنا له ونحن ننشر الطبعة الثانية من كتابنا الفلسفة
اللغوية لأنّ موضوعه تابع لموضوعنا. أو هي خطوة ثانية في تاريخ اللغة باعتبار
منشأها وتكونها ونموها. فالفلسفة اللغوية تبحث في كيف نطق الانسان الأول، وكيف
نشأت اللغة وتولّدت الألفاظ من حكاية الأصوات الخارجية، كقصف الرعد، وهبوب الرياح،
والقطع والكسر، وحكاية التف والنفخ والصفير ونحوها. ومن المقاطع الطبيعية التي
ينطق بها الانسان غريزيا كالتأوه، والزفير. وكيف تنوّعت تلك الأصوات لفظا ومعنى
بالنحت، والابدال، والقلب، حتى صارت ألفاظا مستقلة وتكوّنت الأفعال، والأسماء،
والحروف وصارت اللغة على نحو ما هي عليه.  وأما تاريخ اللغة فيتناول النظر في
ألفاظها وتراكيبها، بعد تمام تكونها، فيبحث فيما طرأ عليهما من التغيير والتجدد أو
الدثور، فيبين الألفاظ والتراكيب التي دثرت من اللغة بالاستعمال، وما قام مقامها
من الألفاظ الجديدة، والتراكيب الجديدة، بما تولّد فيها، أو اقتبسته من سواها، مع
بيان الأحوال التي قضت بدثور القديم وتولد الجديد، وأمثلة مما دثر، أو أهمل، أو
تولّد، أو دخل.%
}

\pardir TRT\textdir TRT
\arabictext

\end{document}
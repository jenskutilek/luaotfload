\pdfvariable suppressoptionalinfo \numexpr1+32+64+128+512\relax

\documentclass{article}

\usepackage{luatexbase}
\usepackage{setspace}
\usepackage{fontspec}
\directlua{dofile("harf-luaotfload.lua")}
\defaultfontfeatures{RawFeature={mode=harf}}

\setmainfont{Amiri}[Script=Arabic]

\begin{document}
\begingroup\pardir TRT\textdir TRT
\setstretch{1.8}
هذا كتاب صغير في بحث جديد، تنبّهنا له ونحن ننشر الطبعة الثانية من كتابنا الفلسفة
اللغوية لأنّ موضوعه تابع لموضوعنا. أو هي خطوة ثانية في تاريخ اللغة باعتبار
منشأها وتكونها ونموها. فالفلسفة اللغوية تبحث في كيف نطق الانسان الأول، وكيف
نشأت اللغة وتولّدت الألفاظ من حكاية الأصوات الخارجية، ك\textbf{قصف الرعد}، وهبوب الرياح،
والقطع والكسر، وحكاية التف والنفخ والصفير ونحوها. ومن المقاطع الطبيعية التي
ينطق بها الانسان غريزيا كالتأوه، والزفير. وكيف تنوّعت تلك الأصوات لفظا ومعنى
بالنحت، والابدال، والقلب، حتى صارت ألفاظا مستقلة وتكوّنت الأفعال، والأسماء،
والحروف وصارت اللغة على نحو ما هي عليه.  وأما تاريخ اللغة فيتناول النظر في
ألفاظها وتراكيبها، بعد تمام تكونها، فيبحث فيما طرأ عليهما من التغيير والتجدد أو
الدثور، فيبين الألفاظ والتراكيب التي دثرت من اللغة بالاستعمال، وما قام مقامها
من الألفاظ الجديدة، والتراكيب الجديدة، بما تولّد فيها، أو اقتبسته من سواها، مع
بيان الأحوال التي قضت بدثور القديم وتولد الجديد، وأمثلة مما دثر، أو أهمل، أو
تولّد، أو دخل.

المحتوى الذي تصل إليه (في حالة عدم استخدام مواقع مؤمنة عن طريق بروتوكول نقل
النص التشعبي الآمن {\textdir TLT HTTPS}).
\par\endgroup\newpage

\def\arabictext{%
هذا كتاب صغير في بحث جديد، تنبّهنا له ونحن ننشر الطبعة الثانية من كتابنا الفلسفة
اللغوية لأنّ موضوعه تابع لموضوعنا. أو هي خطوة ثانية في تاريخ اللغة باعتبار
منشأها وتكونها ونموها. فالفلسفة اللغوية تبحث في كيف نطق الانسان الأول، وكيف
نشأت اللغة وتولّدت الألفاظ من حكاية الأصوات الخارجية، كقصف الرعد، وهبوب الرياح،
والقطع والكسر، وحكاية التف والنفخ والصفير ونحوها. ومن المقاطع الطبيعية التي
ينطق بها الانسان غريزيا كالتأوه، والزفير. وكيف تنوّعت تلك الأصوات لفظا ومعنى
بالنحت، والابدال، والقلب، حتى صارت ألفاظا مستقلة وتكوّنت الأفعال، والأسماء،
والحروف وصارت اللغة على نحو ما هي عليه.  وأما تاريخ اللغة فيتناول النظر في
ألفاظها وتراكيبها، بعد تمام تكونها، فيبحث فيما طرأ عليهما من التغيير والتجدد أو
الدثور، فيبين الألفاظ والتراكيب التي دثرت من اللغة بالاستعمال، وما قام مقامها
من الألفاظ الجديدة، والتراكيب الجديدة، بما تولّد فيها، أو اقتبسته من سواها، مع
بيان الأحوال التي قضت بدثور القديم وتولد الجديد، وأمثلة مما دثر، أو أهمل، أو
تولّد، أو دخل.%
}

\begingroup\pardir TRT\textdir TRT
\fontspec{Amiri}[Script=Arabic, Scale=1.5]
\setstretch{2.4}
\arabictext
\par\endgroup\newpage

\begingroup\pardir TRT\textdir TRT
\fontspec{Aref Ruqaa}[Script=Arabic, Scale=1.2]
\setstretch{2.4}
\arabictext
\par\endgroup\newpage

\begingroup\pardir TRT\textdir TRT
\fontspec{Noto Nastaliq Urdu}[Script=Arabic]
\setstretch{2.5}
\arabictext
\par\endgroup\newpage

\begingroup\pardir TRT\textdir TRT
\fontspec{Amiri}[Script=Arabic, Scale=2]
قلبي {\addfontfeature{Color=FF0000} تلون كله} بلون الحب.
\par\endgroup\newpage

\begingroup\pardir TRT\textdir TRT
\fontspec{Amiri Quran Colored}[Script=Arabic, Scale=2]
\setstretch{3.8}
\parindent=0pt
\leftskip=0pt plus 1fil
\rightskip=0pt plus -1fil
\parfillskip=0pt plus 2fil
\def\aya#1{{\textdir TLT ^^^^06dd#1}}
^^^^fdfd~\aya{١}\par
ٱلۡحَمۡدُ لِلَّهِ رَبِّ ٱلۡعَٰلَمِینَ~\aya{٢} ٱلرَّحۡمَٰنِ ٱلرَّحِیمِ~\aya{٣} مَٰلِكِ یَوۡمِ ٱلدِّینِ~\aya{٤} إِیَّاكَ نَعۡبُدُ وَإِیَّاكَ نَسۡتَعِینُ~\aya{٥} ٱهۡدِنَا ٱلصِّرَٰطَ ٱلۡمُسۡتَقِیمَ~\aya{٦} صِرَٰطَ ٱلَّذِینَ أَنۡعَمۡتَ عَلَیۡهِمۡ غَیۡرِ ٱلۡمَغۡضُوبِ عَلَیۡهِمۡ وَلَا ٱلضَّاۤلِّینَ~\aya{٧}\par
\endgroup\newpage

\begingroup
\fontspec{BungeeColor1.ttf}[Script=Latin, RawFeature={colr=true}, Scale=5]
ABCDEFG\par
\fontspec{BungeeColor1.ttf}[Script=Latin, RawFeature={colr=2}, Scale=5]
ABCDEFG\par

\fontspec{Noto Color Emoji}[Scale=2]
💙💚💛💜💝🖤🧡😀😇🧔🏽🦆
\endgroup\newpage

\begingroup
\TeX\ = τεχ in Greek

(\textit{f\/})

\def\l#1#2{%
  \begingroup%
  \fontspec{amiri-regular.ttf}[Script=Arabic, Language=#1]
  #2%
  \endgroup%
}
\l{Arabic}  {٠١٢٣٤٥٦٧٨٩}\par
\l{Arabic}  {۰۱۲۳۴۵۶۷۸۹}\par
\l{Urdu}    {۰۱۲۳۴۵۶۷۸۹}\par
\l{Sindhi}  {۰۱۲۳۴۵۶۷۸۹}\par
\l{Kashmiri}{۰۱۲۳۴۵۶۷۸۹}\par

Some text then inline math \(E=mc^2\) then \(\hbox{text inside inline math}\), then
display math \[E=mc^2\hbox{ with text inside}\]

\begingroup
% Amiri does not have “onum” feature, so this produces a warning, which is what
% we are testing here.
\fontspec{Amiri}[Script=Latin, Numbers=OldStyle]
Hello 0123456789
\endgroup

\endgroup\newpage

\begingroup
\parindent=0pt
\fontspec{NotoSerifCJK-Regular.ttc}[Script=Latin]

\noindent
\smallskip
offbeat office baffle coffee HAVANA\par
\smallskip
\begingroup
\rightskip=0pt plus1fil \pretolerance=-1 \hyphenpenalty=-10000
offbeat office baffle coffee HAVANA\par
\endgroup
\endgroup
\newpage

\newfontfamily\testr{lmroman10-regular.otf}[Script=Latin, Ligatures=Discretionary]
\begingroup
\parindent=0pt
\testr
\noindent
\smallskip
offbeat office baffle {\rm baffle} coffee HAVANA\par
\smallskip
\begingroup
\rightskip=0pt plus1fil \pretolerance=-1 \hyphenpenalty=-10000
offbeat office baffle {\rm baffle} coffee HAVANA\par
\endgroup
\endgroup
\newpage

\newfontfamily\testb{lmroman10-bold.otf}[Script=Latin]
\newfontfamily\testi{lmromanslant10-regular.otf}[Script=Latin]
\newfontfamily\testR{lmroman12-regular.otf}[Script=Latin, Scale=1.2]
\testr
\hrule
\vskip 1in
\centerline{\testb A SHORT STORY}
\vskip 6pt
\centerline{\testi by A. U. Thor}
\vskip .5cm
Once upon a time, in a distant {\testR galaxy called} Ööç, there lived a computer named
R.~J. Drofnats.

Mr.~Drofnats-----or ‘‘R. J.,’’ as he {\textdir TRT preferred} to be called-----was happiest when he
was at work typesetting beautiful documents.
\vskip 1in
\hrule
\newpage

\font\test=cmr10 at 12pt\test
\input story
\end{document}

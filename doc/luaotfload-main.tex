%% Copyright (C) 2009-2018
%%
%%      by  Elie Roux      <elie.roux@telecom-bretagne.eu>
%%      and Khaled Hosny   <khaledhosny@eglug.org>
%%      and Philipp Gesang <phg@phi-gamma.net>
%%
%% This file is part of Luaotfload.
%%
%%      Home:      https://github.com/lualatex/luaotfload
%%      Support:   <lualatex-dev@tug.org>.
%%
%% Luaotfload is under the GPL v2.0 (exactly) license.
%%
%% ----------------------------------------------------------------------------
%%
%% Luaotfload is free software; you can redistribute it and/or
%% modify it under the terms of the GNU General Public License
%% as published by the Free Software Foundation; version 2
%% of the License.
%%
%% Luaotfload is distributed in the hope that it will be useful,
%% but WITHOUT ANY WARRANTY; without even the implied warranty of
%% MERCHANTABILITY or FITNESS FOR A PARTICULAR PURPOSE. See the
%% GNU General Public License for more details.
%%
%% You should have received a copy of the GNU General Public License
%% along with Luaotfload; if not, see <http://www.gnu.org/licenses/>.
%%
%% ----------------------------------------------------------------------------
%%

\beginfrontmatter

  \setdocumenttitle  {The \identifier{luaotfload} package}
  \setdocumentdate   {2019-12-23 v3.1202-dev}
  \setdocumentauthor {LaTeX3 Project\\
                      Elie Roux · Khaled Hosny · Philipp Gesang · Ulrike Fischer · Marcel Krüger\\
                      Home: \hyperlink {https://github.com/latex3/luaotfload}}

  \typesetdocumenttitle

  \beginabstractcontent
    This package is an adaptation of the \CONTEXT font loading system.
    It allows for loading \OpenType fonts with an extended syntax and adds
    support for a variety of font features.

    After discussion of the font loading API, this manual gives an
    overview of the core components of \identifier{Luaotfload}: The
    packaged font loader code, the names database, configuration, and
    helper functions on the \LUA\ end.
  \endabstractcontent

\endfrontmatter

\pdfbookmark[1]{\contentsname}{table}
\typesetcontent
\beginsection {Engine and version support}
\identifier{luaotfload} is a quite large and complex package. It imports code from context which is actively developed along with the luatex binary. It is not possible
to support a large number of engines variants or versions.

Supported is the \identifier{luatex} versions of a current TeXLive 2019 (and a current MiKTeX). Beginning with version 3.1 of this package also \identifier{luahbtex} is supported.
\endsection

\beginsection{Changes}
\beginsubsection {New in version 3.12 (by Ulrike Fischer/Marcel Krüger)}
\begin{itemize}
\item Corrected a number of small bugs in harf mode.
\item  Extension\marginpar{Experimental!} of the \identifier{color} key to allow
coloring of specific output glyphs, see page~\pageref{color-glyphs}
\item  A\marginpar{Experimental!} new \identifier{fallback} key to allow to define fallback fonts, see page~\pageref{fallback}
\item  A\marginpar{Experimental!} new \identifier{multiscript} key to allow to use a font
for more than one script, see page~\pageref{multiscript}
\end{itemize}

\beginsubsection {New in version 3.11 (by Ulrike Fischer/Marcel Krüger)}
\begin{itemize}
\item Changed the handling of the \identifier{script} key in harf mode to be more compatible with behaviour of the node mode. It now expects the name of a script that is actually in the font instead of a ISO 15924 script tag. See issue 117.
\item Corrected a number of small typos and bugs in harf mode.
\end{itemize}
\beginsubsection {New in version 3.1 (by Ulrike Fischer/Marcel Krüger)}
\begin{itemize}
\item   The package has been moved to the github of the LaTeX3 Project and is now maintained
        officially by the LaTeX3 Project team.
\item   Code to use the harfbuzz library of luahbtex has been added. See the description of the harf mode.
\item   fonts in ttc-collections can now be indexed by name.
\item   To reduce the polution of the global lua enviroment a number of lua tables have been removed.
        Only the tables \identifier{luaotfload}, \identifier{fonts} and \identifier{nodes} have been kept there.
\item   The fontloader has been synched with the context files from 2019-10-29.

\end{itemize}
\endsubsection


\beginsubsection {New in version 3.00 (by Ulrike Fischer/Marcel Krüger)}
\begin{itemize}
\item Default Ignorable characters are now invisible by default (issue 63). This can be deactivated with the option \texttt{invisible}.
\end{itemize}
\endsubsection

\beginsubsection {New in version 2.99 (by Ulrike Fischer)}
\begin{itemize}
\item Code cleanup.
\item The fontloader has been synched with the context files from 2019-08-11.
\end{itemize}
\endsubsection
\beginsubsection {New in version 2.98 (by Ulrike Fischer)}
\begin{itemize}
\item The\marginpar{\mbox{}\hfill \textbf{breaking change!}} handling of missing chars has been changed. In This version a missing char will insert the \inlinecode{/.notdef} char of the fonts (this is sometimes a space, sometimes a rectangle with a cross) and no longer simply ignore the glyph. This behaviour can be reverted by using \inlinecode{notdef=false} as font feature.
\item The font feature \inlinecode{embolden} can now be used to fake a bold font.
\item The fontloader has been synched with the context files from 2019-07-04.
\end{itemize}
\endsubsection
\beginsubsection {New in version 2.97 (by Ulrike Fischer)}
\begin{itemize}
\item the new generic fontloader improves the handling of large fonts (but some fonts still need a 64bit luatex version to create the font files).
\item A number of small bug (also in luaotfload-tool) have been corrected, see the NEWS file for details.

\end{itemize}
\endsubsection

\beginsubsection {New in version 2.96 (by Ulrike Fischer)}
\begin{itemize}
\item In\marginpar{\mbox{}\hfill \textbf{Incompatible change!}}
version 2.95 letterspacing was broken due to a change in the fontloader (issue 38). This has been repaired. At the same time a number of oddities and bugs in the letterspacing has been corrected. This can change existing documents. See page~\pageref{p:letterspace} for more information.

\item A problem with the detection of bold fonts has been corrected (issue 41, pull request 42).

\end{itemize}
\endsubsection



\beginsubsection {New in version 2.95 (by Ulrike Fischer)}
\begin{itemize}
\item
This version imports from context the generic fontloader in the version of 2019-01-28. Contrary to the last announcement, it still works with luatex 1.07. So updates will continue.

\item The handling of the lucida-fonts had been improved (issue 33).

\item tex-files are no longer misused as font fallbacks (issue 35).

\item The resolver code has be refactorated (pull request 36).

\end{itemize}
\endsubsection
\beginsubsection {New in version 2.94 (by Ulrike Fischer)}
\begin{itemize}
\item
This version imports from context the generic fontloader in the version of 2018-12-19. It is the last version that works with luatex 1.07 and texlive 2018. As context has moved to luatex 1.09 newer versions of the fontloader needs now this luatex version too. This means that until the texlive 2018 freeze there will be probably no update of luaotfload.

\item This version changes the handling of the \inlinecode{mode} key. It no longer accepts only the values \inlinecode{base} and \inlinecode{node}, but can be used to load a font with an alternative font loader/renderer.


\end{itemize}
\endsubsection

\beginsubsection {New in version 2.93 (by Ulrike Fischer)}
Mainly internal clean up of the version info to allow automatic versioning.
\endsubsection

\beginsubsection {New in version 2.92 (by Ulrike Fischer)}

\begin{itemize}

  \item Better devanagari support (issue \#9).
  \item \identifier{Luaotfload} doesn't work when luatex is used with the option \inlinecode{--safer}. So it now aborts cleanly when the option is detected -- but you still can get errors from fontspec later! (issue \#12).
  \item  The syntax \inlinecode{file:} for legacy font works again (issue \#11).
  \item The fontloader has been synched with the newest context version from october, 18.
\end{itemize}
\endsubsection
\beginsubsection {New in version 2.91 (by Ulrike Fischer)}

This version mostly correct two bugs found in the previous fontloader: Glyphvariants weren't copied and pasted correctly. Glyphs encoded in the PUA couldn't be accessed anymore.

\endsubsection


\beginsubsection {New in version 2.9 (by Ulrike Fischer)}

On the one side there is not very much new in this version: The native components of \identifier{Luaotfload} are nearly unchanged. A few bugs have been corrected, the various files lists which loads the components of the font loader have been cleaned up.

On the other side there is a lot new:

\begindescriptions

  \beginaltitem {Fontloader} The fontloader files imported from \CONTEXT\ have been updated to the current version.
   This was necessary to make \identifier{Luaotfload} compatible with the coming \LUATEX 1.08/1.09. Compared to the previous version from february 2017 quite a number of things have changed. Most importantly the handling of arabic fonts has greatly improved. But this also means that changes in the output are possible.
  \endaltitem

  \beginaltitem {Lualibs} The update of the fontloader files also required an update of the \identifier{Lualibs} package. This \identifier{Luaotfload} version needs version 2.6 of \identifier{Lualibs}.
  \endaltitem

  \beginaltitem {Maintenance} As the current maintainer wasn't available and it was urgent to get a \identifier{Luaotfload} compatible with \LUATEX 1.08/1.09 maintenance has been transfered to Ulrike Fischer and Marcel Krüger. The package was maintained and developed at \hyperlink{https://github.com/u-fischer/luaotfload}. Issues should be reported there.
  \endaltitem

  \beginaltitem {Documentation}
  The core of documentation is nearly unchanged. I added this introduction. I recreated with the help of @marmot the graphic on \pageref{file-graph}. I updated the file lists. I imported as appendix pdf versions of the two man files which are part of the \identifier{Luaotfload} documentation.
  \endaltitem

\enddescriptions

\endsubsection
\endsection
%%%%%%%%%%%%%%%%%%%%%%%%%%%%%%%%%%%%%%%%%%%%%%%%%%%%%%%%%%%%%%%%%%%%%%%%%%%%%%%
\beginsection {Introduction}
%%%%%%%%%%%%%%%%%%%%%%%%%%%%%%%%%%%%%%%%%%%%%%%%%%%%%%%%%%%%%%%%%%%%%%%%%%%%%%%

Font management and installation has always been painful with \TEX.  A
lot of files are needed for one font (\abbrev{tfm}, \abbrev{pfb},
\abbrev{map}, \abbrev{fd}, \abbrev{vf}), and due to the 8-Bit encoding
each font is limited to 256 characters.

But the font world has evolved since the original \TEX, and new
typographic systems have appeared, most notably the so called
\emphasis{smart font} technologies like \OpenType fonts (\abbrev{otf}).

These fonts can contain many more characters than \TEX fonts, as well
as additional functionality like ligatures, old-style numbers, small
capitals, etc., and support more complex writing systems like Arabic
and Indic\footnote{%
  Unfortunately, the default fontloader of \identifier{luaotfload} doesn‘t support many Indic
  scripts correctly. For these scripts it is recommended to use the harf mode along with the binary \identifier{luahbtex}.}
scripts.

\OpenType fonts are widely deployed and available for all modern
operating systems.

As of 2013 they have become the de facto standard for advanced text
layout.

However, until recently the only way to use them directly in the \TEX
world was with the \XETEX engine.

Unlike \XETEX, \LUATEX has no built-in support for \OpenType or
technologies other than the original \TEX fonts.

Instead, it provides hooks for executing \LUA code during the \TEX run
that allow implementing extensions for loading fonts and manipulating
how input text is processed without modifying the underlying engine.

This is where \identifier{luaotfload} comes into play:
Based on code from \CONTEXT, it extends \LUATEX with functionality necessary
for handling \OpenType fonts.

Additionally, it provides means for accessing fonts known to the operating
system conveniently by indexing the metadata.

\endsection

%%%%%%%%%%%%%%%%%%%%%%%%%%%%%%%%%%%%%%%%%%%%%%%%%%%%%%%%%%%%%%%%%%%%%%%%%%%%%%%
\beginsection {Thanks}
%%%%%%%%%%%%%%%%%%%%%%%%%%%%%%%%%%%%%%%%%%%%%%%%%%%%%%%%%%%%%%%%%%%%%%%%%%%%%%%

\identifier{Luaotfload} is part of \LUALATEX, the community-driven
project to provide a foundation for using the \LATEX format with the
full capabilites of the \LUATEX engine.
%
As such, the distinction between end users, contributors, and project
maintainers is intentionally kept less strict, lest we unduly
personalize the common effort.

Nevertheless, the current maintainers would like to express their
gratitude to Khaled Hosny, Akira Kakuto, Hironori Kitagawa and Dohyun
Kim.
%
Their contributions -- be it patches, advice, or systematic
testing -- made the switch from version 1.x to 2.2 possible.
%
Also, Hans Hagen, the author of the font loader, made porting the
code to \LATEX a breeze due to the extra effort he invested into
isolating it from the rest of \CONTEXT, not to mention his assistance
in the task and willingness to respond to our suggestions.

\endsection

%%%%%%%%%%%%%%%%%%%%%%%%%%%%%%%%%%%%%%%%%%%%%%%%%%%%%%%%%%%%%%%%%%%%%%%%%%%%%%%
\beginsection {Loading Fonts}
%%%%%%%%%%%%%%%%%%%%%%%%%%%%%%%%%%%%%%%%%%%%%%%%%%%%%%%%%%%%%%%%%%%%%%%%%%%%%%%

\identifier{luaotfload} supports an extended font request syntax:

\beginnarrower
      \nonproportional{\string\font\string\foo\space= \string{}%
      \meta{prefix}\nonproportional{:}%
      \meta{font name}\nonproportional{:}%
      \meta{font features}\nonproportional{\string}}%
      \meta{\TEX font features}
\endnarrower

\noindent
The curly brackets are optional and escape the spaces in the enclosed
font name.
%
Alternatively, double quotes serve the same purpose.
%
A selection of individual parts of the syntax are discussed below;
for a more formal description see figure \ref{font-syntax}.

\beginsyntaxfloat
  {font-syntax}
  {Font request syntax.
   Braces or double quotes around the
   \emphasis{specification} rule will
   preserve whitespace in file names.
   In addition to the font style modifiers
   (\emphasis{slash-notation}) given above, there
   are others that are recognized but will be silently
   ignored: \nonproportional{aat},
            \nonproportional{icu}, and
            \nonproportional{gr}.
   The special terminals are:
   \smallcaps {feature\textunderscore id} for a valid font
      feature name and
   \smallcaps {feature\textunderscore value} for the corresponding
      value.
   \smallcaps {tfmname} is the name of a \abbrev{tfm} file.
   \smallcaps {digit}  again refers to bytes 48--57, and
   \smallcaps {all\textunderscore characters} to all byte values.
   \smallcaps {csname} and \smallcaps {dimension} are the \TEX concepts.}
%
      <definition>      ::= `\\font', {\sc csname}, `=', <font request>, [ <size> ] ;

      <size>            ::= `at', {\sc dimension} ;

      <font request>    ::= `"', <unquoted font request> `"'
      \alt                  `{', <unquoted font request> `}'
      \alt                  <unquoted font request> ;

      <unquoted font request> ::= <specification>, [`:', <feature list> ]
      \alt                        <path lookup>, [ [`:'], <feature list> ] ;

      <specification>    ::= <prefixed spec>, [ <subfont no> ], \{ <modifier> \}
      \alt                   <anon lookup>, \{ <modifier> \} ;

      <prefixed spec>    ::= `combo:', <combo list>
      \alt                   `file:', <file lookup>
      \alt                   `name:', <name lookup> ;

      <combo list>       ::= <combo def 1>, \{ `;', <combo def>  \} ;

      <combo def 1>      ::= <combo id>, `->', <combo id> ;

      <combo def>        ::= <combo id>, `->', <combo id chars> ;

      <combo id>         ::= (`(', \{ {\sc digit} \}, `)' | \{ {\sc digit} \} ) ;

      <combo id chars>   ::= (`(', \{ {\sc digit} \}, `,', <combo chars>, `)'
      \alt                   \{ {\sc digit} \} ) ;

      <combo chars>      ::= `fallback'
      \alt                   \{ <combo range>, \{ `*', <combo range> \} \} ;

      <combo range>      ::= <combo num>, [ `-', <combo num> ] ;

      <combo num>        ::= `0x', \{ {\sc hexdigit}  \}
      \alt                   `U+', \{ {\sc digit} \}
      \alt                   \{ {\sc digit} \} ;

      <file lookup>      ::= \{ <name character> \} ;

      <name lookup>      ::= \{ <name character> \} ;

      <anon lookup>      ::= {\sc tfmname} | <name lookup> ;

      <path lookup>      ::= `[', \{ <path content> \}, `]', [ <subfont no> ] ;

      <path content>     ::= <path balanced>
      \alt                   `\\', {\sc all_characters}
      \alt                   {\sc all_characters} - `]'

      <path balanced>    ::= `[', [ <path content> ], `]'

      <modifier>         ::= `/', (`I' | `B' | `BI' | `IB' | `S=', \{ {\sc digit} \} ) ;

      <subfont no>       ::= `(', \{ {\sc digit} \}, `)' ;

      <feature list>     ::= <feature expr>, \{ `;', <feature expr> \} ;

      <feature expr>     ::= {\sc feature_id}, `=', {\sc feature_value}
      \alt                   <feature switch>, {\sc feature_id} ;

      <feature switch>   ::= `+' | `-' ;

      <name character>   ::= {\sc all_characters} - ( `(' | `/' | `:' ) ;
\endsyntaxfloat

%% Below guarded space gets borked in index; why‽
\beginsubsection{Prefix -- the \texorpdfstring{\identifier{luaotfload}}{luaotfload}{ }Way}

In \identifier{luaotfload}, the canonical syntax for font requests
requires a \emphasis{prefix}:
%
\beginnarrower
  \nonproportional{\string\font\string\fontname\space= }%
  \meta{prefix}%
  \nonproportional{:}%
  \meta{fontname}%
  \dots
\endnarrower
%
where \meta{prefix} is either \inlinecode{file:} or \inlinecode {name:}.\footnote{%
  \identifier{Luaotfload} also knows two further prefixes, \inlinecode {kpse:}
  and \inlinecode {my:}.
  %
  A \inlinecode {kpse} lookup is restricted to files that can be found by
  \identifier{kpathsea} and will not attempt to locate system fonts.
  %
  This behavior can be of value when an extra degree of encapsulation is
  needed, for instance when supplying a customized tex distribution.

  The \inlinecode {my} lookup takes this a step further: it lets you define
  a custom resolver function and hook it into the \luaident{resolve_font}
  callback.
  %
  This ensures full control over how a file is located.
  %
  For a working example see the
  \hyperlink [test in the luaotfload
  repo]{https://github.com/latex3/luaotfload/blob/master/testfiles/my-resolver.lvt}.
}
%
It determines whether the font loader should interpret the request as
a \emphasis{file name} or
  \emphasis{font name}, respectively,
which again influences how it will attempt to locate the font.
%
Examples for font names are
            “Latin Modern Italic”,
            “GFS Bodoni Rg”, and
            “PT Serif Caption”
-- they are the human readable identifiers
usually listed in drop-down menus and the like.\footnote{%
  Font names may appear like a great choice at first because they
  offer seemingly more intuitive identifiers in comparison to arguably
  cryptic file names:
  %
  “PT Sans Bold” is a lot more descriptive than \fileent{PTS75F.ttf}.
  On the other hand, font names are quite arbitrary and there is no
  universal method to determine their meaning.
  %
  While \identifier{luaotfload} provides fairly sophisticated heuristic
  to figure out a matching font style, weight, and optical size, it
  cannot be relied upon to work satisfactorily for all font files.
  %
  For an in-depth analysis of the situation and how broken font names
  are, please refer to
  \hyperlink [this post]{http://www.ntg.nl/pipermail/ntg-context/2013/073889.html}
  by Hans Hagen, the author of the font loader.
  %
  If in doubt, use filenames.
  %
  \fileent{luaotfload-tool} can perform the matching for you with the
  option \inlinecode {--find=<name>}, and you can use the file name it returns
  in your font definition.
}
%
In order for fonts installed both in system locations and in your
\fileent{texmf} to be accessible by font name, \identifier{luaotfload} must
first collect the metadata included in the files.
%
Please refer to section~\ref{sec:fontdb} below for instructions on how to
create the database.

File names are whatever your file system allows them to be, except
that that they may not contain the characters
  \inlinecode {(},
  \inlinecode {:}, and
  \inlinecode {/}.
%
As is obvious from the last exception, the \inlinecode {file:} lookup will
not process paths to the font location -- only those
files found when generating the database are addressable this way.
%
Continue below in the \XETEX section if you need to load your fonts
by path.
%
The file names corresponding to the example font names above are
  \fileent{lmroman12-italic.otf},
  \fileent{GFSBodoni.otf}, and
  \fileent{PTZ56F.ttf}.

\endsubsection

\beginsubsection {Bracketed Lookups}
\label{sec:bracket}
Bracketed lookups allow for arbitrary character content to be used in a
definition.
%
A simple bracketed request looks follows the scheme

\beginnarrower
  \nonproportional{\string\font\string\fontname\space = [}%
  \meta{/path/to/file}%
  \nonproportional{]}
\endnarrower

\noindent
Inside the square brackets, every character except for a closing bracket is
permitted, allowing for  arbitrary paths to a font file -- including Windows
style paths with UNC or drive letter prepended -- to be specified.
%
The \identifier{Luaotfload} syntax differs from \XETEX in that the subfont
selector goes \emphasis{after} the closing bracket:

\beginnarrower
  \nonproportional{\string\font\string\fontname\space = [}%
  \meta{/path/to/file}%
  \nonproportional{]}
  \nonproportional{(}n\nonproportional{)}
\endnarrower

Naturally, path-less file names are equally valid and processed the
same way as an ordinary \inlinecode {file:} lookup.

\beginsubsection {Compatibility}

In addition to the regular prefixed requests, \identifier{luaotfload}
accepts loading fonts the \XETEX way.
%
There are again two modes: bracketed and unbracketed.
For the bracketed variety, see above, \ref{sec:bracket}.

Unbracketed (or, for lack of a better word: \emphasis{anonymous})
font requests resemble the conventional \TEX syntax.

\beginnarrower
  \nonproportional{\string\font\string\fontname\space= }%
  \meta{font name}
  \dots
\endnarrower
\endsubsection

However, they have a broader spectrum of possible interpretations:
before anything else, \identifier{luaotfload} attempts to load a
traditional \TEX Font Metric (\abbrev{tfm} or \abbrev{ofm}).
%
If this fails, it performs a \inlinecode {path:} lookup, which itself will
fall back to a \inlinecode {file:} lookup.
%
Lastly, if none of the above succeeded, attempt to resolve the request as a
\inlinecode {name:} lookup by searching the font index for \meta{font name}.
%
The behavior of this “anonymous” lookup is configurable, see the configuation
manpage for details.

Furthermore, \identifier{luaotfload} supports the slashed (shorthand)
font style notation from \XETEX.

\beginnarrower
  \nonproportional{\string\font\string\fontname\space= }%
  \meta{font name}%
  \nonproportional{/}%
  \meta{modifier}
  \dots
\endnarrower

\noindent
Currently, four style modifiers are supported:
  \inlinecode {I} for italic shape,
  \inlinecode {B} for bold   weight,
  \inlinecode {BI} or \inlinecode {IB} for the combination of both.
%
Other “slashed” modifiers are too specific to the \XETEX engine and
have no meaning in \LUATEX.

\endsubsection

\beginsubsection{Examples}

\beginsubsubsection{Loading by File Name}

For example, conventional \TeX\ font can be loaded with a
\inlinecode {file:} request like so:

\beginlisting
  \font \lmromanten = {file:ec-lmr10} at 10pt
\endlisting

The \OpenType version of Janusz Nowacki’s font \emphasis{Antykwa
Półtawskiego}\footnote{%
  \hyperlink {http://jmn.pl/antykwa-poltawskiego/}, also available in
  in \TEX Live.
}
in its condensed variant can be loaded as follows:

\beginlisting
  \font \apcregular = file:antpoltltcond-regular.otf at 42pt
\endlisting

The next example shows how to load the \emphasis{Porson} font digitized by
the Greek Font Society using \XETEX-style syntax and an absolute path from a
non-standard directory:

\beginlisting
  \font \gfsporson = "[/tmp/GFSPorson.otf]" at 12pt
\endlisting

\identifier{TrueType} collection files (the extension is usually
\inlinecode{.ttc}) contain more than a single font. In order to refer to these
subfonts, the respective index or the respective PostScript font name may be
added in parentheses after the file name.\footnote{%
  Incidentally, this syntactical detail also prevents one from loading files
  that end in balanced parentheses.
}

\beginlisting
  \font \cambriamain = "file:cambria.ttc(0)" at 10pt
  \font \cambriamath = "file:cambria.ttc(1)" at 10pt
  \font \Cambriamain = "file:cambria.ttc(Cambria)" at 10pt
  \font \Cambriamath = "file:cambria.ttc(CambriaMath)" at 10pt
\endlisting

and likewise, requesting subfont inside a TTC container by path:

\beginlisting
  \font \asanamain = "[/home/typesetter/.fonts/math/asana.ttc](0):mode=node;+tlig" at 10pt
  \font \asanamath = "[/home/typesetter/.fonts/math/asana.ttc](1):mode=base" at 10pt
\endlisting

\endsubsubsection

\beginsubsubsection{Loading by Font Name}

The \inlinecode {name:} lookup does not depend on cryptic filenames:

\beginlisting
  \font \pagellaregular = {name:TeX Gyre Pagella} at 9pt
\endlisting

A bit more specific but essentially the same lookup would be:

\beginlisting
  \font \pagellaregular = {name:TeX Gyre Pagella Regular} at 9pt
\endlisting

\noindent
Which fits nicely with the whole set:

\beginlisting
  \font\pagellaregular    = {name:TeX Gyre Pagella Regular}    at 9pt
  \font\pagellaitalic     = {name:TeX Gyre Pagella Italic}     at 9pt
  \font\pagellabold       = {name:TeX Gyre Pagella Bold}       at 9pt
  \font\pagellabolditalic = {name:TeX Gyre Pagella Bolditalic} at 9pt

  {\pagellaregular     foo bar baz\endgraf}
  {\pagellaitalic      foo bar baz\endgraf}
  {\pagellabold        foo bar baz\endgraf}
  {\pagellabolditalic  foo bar baz\endgraf}

  ...
\endlisting

\endsubsubsection

\beginsubsubsection{Modifiers}

If the entire \emphasis{Iwona} family\footnote{%
  \hyperlink {http://jmn.pl/kurier-i-iwona/},
  also in \TEX Live.
}
is installed in some location accessible by \identifier{luaotfload},
the regular shape can be loaded as follows:

\beginlisting
  \font \iwona = Iwona at 20pt
\endlisting

\noindent
To load the most common of the other styles, the slash notation can
be employed as shorthand:

\beginlisting
  \font \iwonaitalic     = Iwona/I    at 20pt
  \font \iwonabold       = Iwona/B    at 20pt
  \font \iwonabolditalic = Iwona/BI   at 20pt
\endlisting

\noindent
which is equivalent to these full names:

\beginlisting
  \font \iwonaitalic     = "Iwona Italic"       at 20pt
  \font \iwonabold       = "Iwona Bold"         at 20pt
  \font \iwonabolditalic = "Iwona BoldItalic"   at 20pt
\endlisting

\endsubsubsection
\endsubsection
\endsection

%%%%%%%%%%%%%%%%%%%%%%%%%%%%%%%%%%%%%%%%%%%%%%%%%%%%%%%%%%%%%%%%%%%%%%%%%%%%%%%
\beginsection {Font features}
%%%%%%%%%%%%%%%%%%%%%%%%%%%%%%%%%%%%%%%%%%%%%%%%%%%%%%%%%%%%%%%%%%%%%%%%%%%%%%%

\emphasis{Font features} are the second to last component in the
general scheme for font requests:

\beginnarrower
  \nonproportional{\string\font\string\foo\space= "}%
  \meta{prefix}%
  \nonproportional{:}%
  \meta{font name}%
  \nonproportional{:}%
  \meta{font features}%
  \meta{\TEX font features}%
  \nonproportional{"}
\endnarrower

\noindent
If style modifiers are present (\XETEX style), they must precede
\meta{font features}.

The element \meta{font features} is a semicolon-separated list of feature
tags\footnote{%
  Cf. \hyperlink {http://www.microsoft.com/typography/otspec/featurelist.htm}.
}
and font options.
%
Prepending a font feature with a \inlinecode{+} (plus sign) enables it,
whereas a \inlinecode{-} (minus) disables it. For instance, the request

\beginlisting
  \font \test = LatinModernRoman:+clig;-kern
\endlisting

\noindent activates contextual ligatures (\inlinecode{clig}) and
disables kerning (\inlinecode{kern}).
%
Alternatively the options \inlinecode{true} or \inlinecode{false} can
be passed to the feature in a key/value expression.
%
The following request has the same meaning as the last one:

\beginlisting
  \font \test = LatinModernRoman:clig=true;kern=false
\endlisting

\noindent
Furthermore, this second syntax is required should a font feature
accept other options besides a true/false switch.
%
For example, \emphasis{stylistic alternates} (\inlinecode{salt}) are
variants of given glyphs.
%
They can be selected either explicitly by supplying the variant
index (starting from one), or randomly by setting the value to,
obviously, \inlinecode{random}.

%% TODO   verify that this actually works with a font that supports
%%        the salt/random feature!\fi
\beginlisting
  \font \librmsaltfirst = LatinModernRoman:salt=1
\endlisting

\beginsubsection {Basic font features}\label{sec:mode}

\begindescriptions

  \beginaltitem {mode}
         \identifier{luaotfload} had three \OpenType processing
         \emphasis{modes}:
         \identifier{base}, \identifier{node} and \identifier{harf}.

         \identifier{base} mode works by mapping \OpenType
         features to traditional \TEX ligature and kerning mechanisms.
         %
         Supporting only non-contextual substitutions and kerning
         pairs, it is the slightly faster, albeit somewhat limited, variant.
         %
         \identifier{node} mode works by processing \TeX’s internal
         node list directly at the \LUA end and supports
         a wider range of \OpenType features.
         %
         The downside is that the intricate operations required for
         \identifier{node} mode may slow down typesetting especially
         with complex fonts and it does not work in math mode.

         By default \identifier{luaotfload} is in \identifier{node}
         mode, and \identifier{base} mode has to be requested where needed,
         e.~g. for math fonts.

         \identifier{harf} mode is new in version 3.1 and needs the new \identifier{luahbtex} engine (the mode is ignored if \identifier{luahbtex} is not used). With it is possible to render a font using the harfbuzz library in-built in the new engine. \identifier{harf} mode improves greatly the rendering of indic and arabic scripts and is highly recommended for such scripts.

         When using \identifier{harf} mode it is required to set also the script correctly.


         \beginlisting
         \font\burmesefont={file:NotoSerifMyanmar-Regular.ttf:mode=harf;script=mym2;}
         \font\devafont={file:NotoSansDevanagari-Regular.ttf:mode=harf;script=dev2;}
         \font\banglafont={file:NotoSansBengali-Regular.ttf:mode=harf;script=ben2;}
         \font\tibetanfont={name:Noto Serif Tibetan:mode=harf;script=tibt;}
         \endlisting

         \includegraphics{scripts-demo}


         It is possible to call other font renderers with the mode key. A simple example with a plain reader can be found at \url{https://github.com/latex3/luaotfload/pull/26#issuecomment-437716326}.

  \endaltitem
  \beginaltitem {shaper} \phantomsection\label{shaper-tag}
  If \identifier{luahbtex} and \identifier{harf} mode are used it is possible to specify a shaper, like \identifier{graphite2} or \identifier{ot} (open type).

  \beginlisting
  \pardir TRT\textdir TRT
  \font\test={file:AwamiNastaliq-Regular.ttf:mode=harf;shaper=ot}
  \endlisting

  \includegraphics{shaper-demo}

  \beginlisting
  \pardir TRT\textdir TRT
  \font\test={file:AwamiNastaliq-Regular.ttf:mode=harf;shaper=graphite2}
  \endlisting

  \includegraphics{shaper-demo-graphite}

  \endaltitem
  \beginaltitem {script} \phantomsection\label{script-tag}
         An \OpenType script tag;\footnote{%
           See \hyperlink {http://www.microsoft.com/typography/otspec/scripttags.htm}
           for a list of valid values.
           %
           For scripts derived from the Latin alphabet the value
           \inlinecode{latn} is good choice.
         }
         the default value is \inlinecode{dflt}.
         %
         Some fonts, including very popular ones by foundries like Adobe,
         do not assign features to the \inlinecode{dflt} script, in
         which case the script needs to be set explicitly.
  \endaltitem

  \beginaltitem {language}
         An \OpenType language system identifier,\footnote{%
           Cf. \hyperlink {http://www.microsoft.com/typography/otspec/languagetags.htm}.
         }
         defaulting to \inlinecode{dflt}.
  \endaltitem

  \beginaltitem {color}
         A font color, defined as a triplet of two-digit hexadecimal
         \abbrev{rgb} values, with an optional fourth value for
         transparency
         (where \inlinecode{00} is completely transparent and
         \inlinecode{FF} is opaque).

         For example, in order to set text in semitransparent red:

         \beginlisting
      \font \test = "Latin Modern Roman:color=FF0000BB"
         \endlisting

         Experimental!\marginpar{\mbox{}\hfill NEW in v3.12!}\phantomsection\label{color-glyphs} The \identifier{color} key has been
         extended to accept a table with color declarations of (output) glyphs. For example

         \beginlisting
         \directlua{
           luaotfload.add_colorscheme("myscheme",
           {
             ["00FFFF30"] = {"default"},
             ["FF0000"] = {"kabeng","ebeng"},
             ["00FF00"] = {"ivowelsignbeng"},
             ["0000FF"] = {369} %% 369 is the GID of "nadarabeng"
           })
           }
          \endlisting

          The keys in such a table are like above RGB colors with an optional transparency setting.
          The values are either lists of glyph names or GID numbers.
          Both types are font dependant! Not every font use the same
          glyph names (or even glyph names at all). GID number are font specific anyway. The GID can be found
          by looking up the \verb+["index"]+ entry in the lua file of a font.

          Such a colorscheme can then be used like this:
          \beginlisting
          \font\test={file:NotoSansBengali-Regular.ttf:mode=harf;script=bng2;color=myscheme}
          \endlisting

          and then would give this output:


          {\font\test={file:NotoSansBengali-Regular.ttf:mode=harf;script=bng2;color=myscheme}\test
           কণ্যা এখন কি করিবে
           \char"0995 \char"09BF
           \char"09A8 \char"09CD \char"09A6
           \char"09CD \char"09B0}

  \endaltitem


   \beginaltitem {embolden}
         A factor, defined as a decimal number.

         For example

         \beginlisting
      \font\test = "Latin Modern Roman:mode=node;embolden=2;"
         \endlisting

      {\font\test= "Latin Modern Roman:mode=node;"\test Dies is not bold.
       \font\test= "Latin Modern Roman:mode=node;embolden=2;" \test Dies is a faked bold font.}

  \endaltitem

  \beginaltitem {kernfactor \& letterspace}\phantomsection\label{p:letterspace}
         Define a font with letterspacing (tracking) enabled.
         %
         In \identifier{luaotfload}, letterspacing is implemented by
         inserting additional kerning between glyphs.

         This approach is derived from and still quite similar to the
         \emphasis{character kerning} (\texmacro{setcharacterkerning} /
         \texmacro{definecharacterkerning} \& al.) functionality of
         Context, see the file \fileent{typo-krn.lua} there.
         %
         The main difference is that \identifier{luaotfload} does not
         use \LUATEX attributes to assign letterspacing to regions,
         but defines virtual letterspaced versions of a font.

         The option \identifier{kernfactor} accepts a numeric value that
         determines the letterspacing factor to be applied to the font
         size.
         %
         E.~g. a kern factor of $0.42$ applied to a $10$ pt font
         results in $4.2$ pt of additional kerning applied to each
         pair of glyphs.
         %

         Spaces\marginpar{\mbox{}\hfill NEW in v2.96!} between words are now stretched too. This is consistent with the \XETEX behaviour (and the amount of stretching should be similar). This
         naturally changes the output of a document.  In case you want the old behaviour back use
         \beginlisting
           \directlua{luaotfload.letterspace.keepwordspacing = true}
         \endlisting

         The difference between both options is obvious:

         \begingroup
         \fontspec{IwonaMedium-Regular.otf}[LetterSpace=60]
          New: hello world

          \directlua{luaotfload.letterspace.keepwordspacing = true}

          Old: hello world

          \directlua{luaotfload.letterspace.keepwordspacing = false}
          \endgroup


         Ligatures\marginpar{\mbox{}\hfill NEW in v2.96!}  are no longer split into their component glyphs.
         This change too make the \identifier{luaotfload} more compatible with \XETEX. It also makes it much easier to activate or deactivate ligature sets in letterspaced fonts.
         If you want to split ligatures, you should deactivate as you would do it with a not-letterspaced font, e.g. with the fontspec \identifier{Ligatures} option, or the low-level \identifier{-liga} and similar.


         {\font \test = "file:Iwona-Regular.otf:mode=base;+liga;+tlig;letterspace=12.5"
          \test With standard ligatures: fi -- ff\par

          \font \test = "file:Iwona-Regular.otf:mode=base;-liga;+tlig;letterspace=12.5"
          \test Only with tlig: fi -- ff \par

          \font \test = "file:Iwona-Regular.otf:mode=base;-liga;letterspace=12.5"
          \test No ligatures: fi -- ff \par
         }

         For compatibility with \XETEX an alternative
         \identifier{letterspace} option is supplied that interprets the
         supplied value as a \emphasis{percentage} of the font size but
         is otherwise identical to \identifier{kernfactor}.
         %
         Consequently, both definitions in below snippet yield the same
         letterspacing width:

         \beginlisting
    \font \iwonakernedA = "file:Iwona-Regular.otf:kernfactor=0.125"
    \font \iwonakernedB = "file:Iwona-Regular.otf:letterspace=12.5"
         \endlisting

         The \identifier{microtype} package uses a special implementation of letterspacing, and the commands \inlinecode{\lsstyle} and \inlinecode{\textls} are  not affected by these changes.

         Setting the ligatures with the font options is the recommended way, to activate or deactivate them. In case of special requirements
         specific pairs of letters and ligatures may be exempt from
         letterspacing by defining the \LUA functions
         \luaident{keeptogether} and \luaident{keepligature},
         respectively, inside the namespace \inlinecode {luaotfload.letterspace}.
         %
         Both functions are called whenever the letterspacing callback
         encounters an appropriate node or set of nodes.
         %
         If they return a true-ish value, no extra kern is inserted at
         the current position.
         %
         \luaident{keeptogether} receives a pair of consecutive
         glyph nodes in order of their appearance in the node list.
         %
         \luaident{keepligature} receives a single node which can be
         analyzed into components.
         %
         (For details refer to the \emphasis{glyph nodes} section in the
         \LUATEX reference manual.)
         %
         The implementation of both functions is left entirely to the
         user.
  \endaltitem

\iffalse
  \startbuffer [printvectors]
  \directlua{inspect(fonts.protrusions.setups.default)
             inspect(fonts.expansions.setups.default)}
  \stopbuffer
\fi

  \beginaltitem {protrusion \& expansion}
         These keys control microtypographic features of the font,
         namely \emphasis{character protrusion} and \emphasis{font
         expansion}.
         %
         Their arguments are names of \LUA tables that contain
         values for the respective features.\footnote{%
            For examples of the table layout please refer to the
            section of the file \fileent{luaotfload-fonts-ext.lua} where the
            default values are defined.
            %
            Alternatively and with loss of information, you can dump
            those tables into your terminal by issuing
            \unless \iffalse
              \beginlisting
 \directlua{inspect(fonts.protrusions.setups.default)
            inspect(fonts.expansions.setups.default)}
              \endlisting
            \else
              \typebuffer [printvectors]
            \fi
            at some point after loading \fileent{luaotfload.sty}.
         }
         %
         For both, only the set \identifier{default} is predefined.

         For example, to define a font with the default
         protrusion vector applied\footnote{%
           You also need to set
               \inlinecode {pdfprotrudechars=2} and
               \inlinecode {pdfadjustspacing=2}
           to activate protrusion and expansion, respectively.
           See the
           \hyperlink [\PDFTEX manual]{http://mirrors.ctan.org/systems/pdftex/manual/pdftex-a.pdf}%
           for details.
         }:

         \beginlisting
      \font \test = LatinModernRoman:protrusion=default
         \endlisting
  \endaltitem

 \beginaltitem {invisible}
         Default Ignorable characters are control characters that should be invisible by default even if the font has glyphs for them. Since version 3.0 luaotfload makes them invisible, this can be switch on and off with the \texttt{invisible}. By default it is on.

         For example

         \beginlisting
         \font\amiri={file:amiri-regular.ttf} at 20pt \amiri
         \char"200Dي\char"200D
         \endlisting

       {\font\amiri={file:amiri-regular.ttf} at 20pt \amiri \char"200Dي\char"200D}

    \beginlisting
         \font\amiri={file:amiri-regular.ttf:-invisible;} at 20pt \amiri
         \char"200Dي\char"200D
         \endlisting

       {\font\amiri={file:amiri-regular.ttf:-invisible;} at 20pt \amiri \char"200Dي\char"200D}


  \endaltitem

  \beginaltitem {multiscript}\phantomsection\label{multiscript}
   In\marginpar{New in 3.12 -- experimental} open type fonts many shaping rules are implemented only for specific scripts and so you get correct typesetting only if the \identifier{script} feature is correctly set. This means that to write a text which uses more than one script you have to declare a font for each script and switch fonts even if the font contains glyphs for all scripts.
   \identifier{multiscript} tries to help here. The feature is experimental and bound to change. Feedback is welcome but you use it at your risk.

   \identifier{multiscript} allows you to declare fonts for various script. The value is either \identifier{auto} described below, or a name which has been previously declared or a combination of both. An example for such a named multiscript could look like this (the colors are only for demonstration):

   \beginlisting
   \directlua{
     luaotfload.add_multiscript
       ("cyrlgrekbeng",
        {
          cyrl = "DejaVuSans:mode=node;script=cyrl;color=FF0000;",
          grek = "texgyreheros:mode=harf;script=grek;color=0000FF;",
          beng = "file:NotoSansBengali-Regular.ttf:mode=harf;script=bng2;color=00FF00;"
        }
       )
      }
   \endlisting

  \identifier{cyrlgrekbeng} is the name of the multiscript (use lower case chars). The keys are ISO language tags (not open type tags!), the values are font declarations.

  The multiscript can then be used in a font like this:

  \beginlisting
   \font\test={name:DejaVuSans:mode=node;multiscript=cyrlgrekbeng;}
  \endlisting

  This would lead to this output:

  {\Large \font\test={name:DejaVuSans:mode=node;multiscript=cyrlgrekbeng;}\test
   „a^^^^0301123!?“  „π^^^^0301123!?“ „a!?“  „Б123!?“ a „\char"0995\char"09BF 123“
  }

  It shows that fonts are switched with the scripts.

  Be aware of the following drawbacks:

  \begin{itemize}
  \item Quite a lot chars can and should be used with more than one script, they belong to the Common or Inherited class. Examples are punctuation chars, digits, accents but also emoji. Currently these chars follow the active script. That's why the digit are all typeset with a different font, the accent over the pi is different to the one over the a, and why the opening quote is sometimes different to the closing quote. It is clear that some tools to force a script (and so a font) locally and globally for such chars are needed.

  \item \identifier{multiscript} doesn't change hyphenation patterns or other language or script related features.

  \item Language packages like babel or polyglossia have code to change the script too which could interfere or clash. This hasn't been tested yet.
  \end{itemize}

  It is possible to use the value \identifier{auto} with \identifier{multiscript}. luaotfload will then switch the script if it detects a char belonging to another script (and if the font support this script). This can be useful for fonts supporting more than one script or when using the \identifier{fallback} key described below.

  It is also possible to combine  \identifier{auto} with a named multiscript with the syntax \identifier{multiscript=auto+name}. The rules of the named multiscript will in such cases take precedence and \identifier{auto} used only for other scripts.



  \endaltitem

   \beginaltitem {fallback}\label{fallback}
   XX\marginpar{New in 3.12 -- experimental}
   \endaltitem
\enddescriptions

\endsubsection

\beginsubsection {Non-standard font features}
\identifier{luaotfload} adds a number of features that are not defined
in the original \OpenType specification, most of them
aiming at emulating the behavior familiar from other \TEX engines.
%
Currently (2014) there are three of them:

\begindescriptions

  \beginaltitem {anum}
          Substitutes the glyphs in the \abbrev{ascii} number range
          with their counterparts from eastern Arabic or Persian,
          depending on the value of \identifier{language}.
  \endaltitem

  \beginaltitem {tlig}
          Applies legacy \TEX ligatures\footnote{%
            These contain the feature set \inlinecode {trep} of earlier
            versions of \identifier{luaotfload}.

            Note to \XETEX users: this is the equivalent of the
            assignment \inlinecode {mapping=text-tex} using \XETEX's input
            remapping feature.
          }:

          \unless \iffalse
            %% Using braced arg syntax with inline code appears to be
            %% impossible within Latex tables -- just ignore the weird
            %% exclamation points below.
            \begintabulate [rlrl]
              \beginrow ``  \newcell  {\inlinecode !``! } \newcell  ''  \newcell  {\inlinecode !''!} \endrow
              \beginrow `   \newcell  {\inlinecode !`!  } \newcell  '   \newcell  {\inlinecode !'! } \endrow
              \beginrow "   \newcell  {\inlinecode !"!  } \newcell  --  \newcell  {\inlinecode !--!} \endrow
              \beginrow --- \newcell  {\inlinecode !---!} \newcell  !`  \newcell  {\inlinecode ?!`?} \endrow
              \beginrow ?`  \newcell  {\inlinecode !?`! } \newcell      \newcell                     \endrow
            \endtabulate
          \else
            %% XXX find a way to wrap these in the tabulate environment
            \startframed [frame=off,width=broad,align=middle]
              \startframed [frame=off,width=\dimexpr(\textwidth/2)]
                \startxtable [align=middle]
                    \startxrow \startxcell ``  \stopxcell \startxcell  \inlinecode {``}  \stopxcell \startxcell  ''  \stopxcell \startxcell  \inlinecode {''}  \stopxcell \stopxrow
                    \startxrow \startxcell `   \stopxcell \startxcell  \inlinecode {`}   \stopxcell \startxcell  '   \stopxcell \startxcell  \inlinecode {'}   \stopxcell \stopxrow
                    \startxrow \startxcell "   \stopxcell \startxcell  \inlinecode {"}   \stopxcell \startxcell  --  \stopxcell \startxcell  \inlinecode {--}  \stopxcell \stopxrow
                    \startxrow \startxcell --- \stopxcell \startxcell  \inlinecode {---} \stopxcell \startxcell  !`  \stopxcell \startxcell  \inlinecode {!`}  \stopxcell \stopxrow
                    \startxrow \startxcell ?`  \stopxcell \startxcell  \inlinecode {?`}  \stopxcell \startxcell      \stopxcell \startxcell                    \stopxcell \stopxrow
                \stopxtable
              \stopframed
            \stopframed
          \fi
  \endaltitem

  \beginaltitem {itlc}
          Computes italic correction values (active by default).
  \endaltitem

\enddescriptions

\endsubsection
\endsection

%%%%%%%%%%%%%%%%%%%%%%%%%%%%%%%%%%%%%%%%%%%%%%%%%%%%%%%%%%%%%%%%%%%%%%%%%%%%%%%
\beginsection {Combining fonts}
%%%%%%%%%%%%%%%%%%%%%%%%%%%%%%%%%%%%%%%%%%%%%%%%%%%%%%%%%%%%%%%%%%%%%%%%%%%%%%%

Version 2.7 and later support combining characters from multiple fonts into a
single virtualized one. This requires that the affected fonts be loaded in
advance as well as a special \emphasis{request syntax}. Furthermore, this
allows to define \emphasis{fallback fonts} to supplement fonts that may lack
certain required glyphs.

Combinations are created by defining a font using the \luaident{combo:} prefix.

\beginsubsection {Fallbacks}

For example, the \identifier{Latin Modern} family of fonts does, as indicated
in the name, not provide Cyrillic glyphs. If Latin script dominates in the copy
with interspersed Cyrillic, a fallback can be created from a similiar looking
font like \identifier{Computer Modern Unicode}, taking advantage of the fact
that it too derives from Knuth’s original \identifier{Computer Modern} series:

\beginlisting
  \input luaotfload.sty
  \font \lm  = file:lmroman10-regular.otf:mode=base
  \font \cmu = file:cmunrm.otf:mode=base
  \font \lmu = "combo: 1->\fontid\lm; 2->\fontid\cmu,fallback"
  \lmu Eh bien, mon prince. Gênes et Lueques ne sont plus que des
       apanages, des поместья, de la famille Buonaparte.
  \bye
\endlisting

As simple as this may look on the first glance, this approach is entirely
inappropriate if more than a couple letters are required from a different font.
Because the combination pulls nothing except the glyph data, all of the
important other information that constitute a proper font -- kerning, styles,
features, and suchlike -- will be missing.

\endsubsection %% Fallbacks

\beginsubsection {Combinations}

Generalizing the idea of a \emphasis{fallback font}, it is also possible to
pick definite sets of glyphs from multiple fonts. On a bad day, for instance,
it may be the sanest choice to start out with \identifier{EB Garamond} italics,
typeset all decimal digits in the bold italics of \identifier{GNU Freefont},
and tone down the punctuation with extra thin glyphs from \identifier{Source
Sans}:

\beginlisting
  \def \feats     {-tlig;-liga;mode=base;-kern}
  \def \fileone   {EBGaramond12-Italic.otf}
  \def \filetwo   {FreeMonoBoldOblique.otf}
  \def \filethree {SourceSansPro-ExtraLight.otf}

  \input luaotfload.sty

  \font \one   = file:\fileone  :\feats
  \font \two   = file:\filetwo  :\feats
  \font \three = file:\filethree:\feats

  \font \onetwothree = "combo:  1 -> \fontid\one;
                                2 -> \fontid\two,   0x30-0x39;
                                3 -> \fontid\three, 0x21*0x3f; "

  {\onetwothree \TeX—0123456789—?!}
  \bye
\endlisting

\noindent Despite the atrocious result, the example demonstrates well the
syntax that is used to specify ranges and fonts. Fonts are being referred to by
their internal index which can be obtained by passing the font command into the
\texmacro{fontid} macro, e. g. \inlinecode{\fontid\one}, after a font has been
defined. The first component of the combination is the base font which will be
extended by the others. It is specified by the index alone.

All further fonts require either the literal \inlinecode{fallback} or a list of
codepoint definitions to be appended after a comma. The elements of this list
again denote either single codepoints like \inlinecode{0x21} (referring to the
exclamation point character) or ranges of codepoints (\inlinecode{0x30-0x39}).
Elements are separated by the \identifier{ASCII} asterisk character
(\inlinecode{*}). The characters referenced in the list will be imported from
the respective font, if available.

\endsubsection %% Combinations

\endsection

%%%%%%%%%%%%%%%%%%%%%%%%%%%%%%%%%%%%%%%%%%%%%%%%%%%%%%%%%%%%%%%%%%%%%%%%%%%%%%%
\beginsection {Font names database}
%%%%%%%%%%%%%%%%%%%%%%%%%%%%%%%%%%%%%%%%%%%%%%%%%%%%%%%%%%%%%%%%%%%%%%%%%%%%%%%

\label{sec:fontdb}

As mentioned above, \identifier{luaotfload} keeps track of which
fonts are available to \LUATEX by means of a \emphasis{database}.
%
This allows referring to fonts not only by explicit filenames but
also by the proper names contained in the metadata which is often
more accessible to humans.\footnote{%
  The tool \hyperlink[\fileent{otfinfo}]{http://www.lcdf.org/type/}
  (comes with \TEX Live), when invoked on a font file with the
  \inlinecode {-i} option, lists the variety of name fields defined for
  it.
}

When \identifier{luaotfload} is asked to load a font by a font name,
it will check if the database exists and load it, or else generate a
fresh one.
%
Should it then fail to locate the font, an update to the database is
performed in case the font has been added to the system only
recently.
%
As soon as the database is updated, the resolver will try
and look up the font again, all without user intervention.
%
The goal is for \identifier{luaotfload} to act in the background and
behave as unobtrusively as possible, while providing a convenient
interface to the fonts installed on the system.

Generating the database for the first time may take a while since it
inspects every font file on your computer.
%
This is particularly noticeable if it occurs during a typesetting run.
In any case, subsequent updates to the database will be quite fast.

\beginsubsection[luaotfload-tool]
                {\fileent{luaotfload-tool}}

It can still be desirable at times to do some of these steps
manually, and without having to compile a document.
%
To this end, \identifier{luaotfload} comes with the utility
\fileent{luaotfload-tool} that offers an interface to the database
functionality.
%
Being a \LUA script, there are two ways to run it:
either make it executable (\inlinecode {chmod +x} on unixoid systems) or
pass it as an argument to \fileent{texlua}.\footnote{%
  Tests by the maintainer show only marginal performance gain by
  running with Luigi Scarso’s
  \hyperlink [\identifier{Luajit\kern-.25ex\TEX}]{https://foundry.supelec.fr/projects/luajittex/},
  which is probably due to the fact that most of the time is spent
  on file system operations.

  \emphasis{Note}:
  On \abbrev{MS} \identifier{Windows} systems, the script can be run
  either by calling the wrapper application
  \fileent{luaotfload-tool.exe} or as
  \inlinecode {texlua.exe luaotfload-tool.lua}.
}
%
Invoked with the argument \inlinecode {--update} it will perform a database
update, scanning for fonts not indexed.

\beginlisting
  luaotfload-tool --update
\endlisting

Adding the \inlinecode {--force} switch will initiate a complete
rebuild of the database.

\beginlisting
  luaotfload-tool --update --force
\endlisting

\endsubsection

\beginsubsection{Search Paths}

\identifier{luaotfload} scans those directories where fonts are
expected to be located on a given system.
%
On a Linux machine it follows the paths listed in the
\identifier{Fontconfig} configuration files;
consult \inlinecode {man 5 fonts.conf} for further information.
%
On \identifier{Windows} systems, the standard location is
\inlinecode {Windows\\Fonts},
%
while \identifier{Mac OS~X} requires a multitude of paths to
be examined.
%
The complete list is is given in table \ref{table-searchpaths}.
Other paths can be specified by setting the environment variable
\inlinecode {OSFONTDIR}.
%
If it is non-empty, then search will be extended to the included
directories.

\tablefloat {table-searchpaths}
  {List of paths searched for each supported operating system.}
  {%
    \unless \iffalse
      \begincentered
        \begintabulate [lp{.5\textwidth}]
          \beginrow
            Windows   \newcell \inlinecode !\% WINDIR\%\\ Fonts!
          \endrow
          \beginrow
            Linux     \newcell \fileent{/usr/local/etc/fonts/fonts.conf} and\hfill\break
                               \fileent{/etc/fonts/fonts.conf}
          \endrow
          \beginrow
            Mac       \newcell \fileent{\textasciitilde/Library/Fonts},\break
                               \fileent{/Library/Fonts},\break
                               \fileent{/System/Library/Fonts}, and\hfill\break
                               \fileent{/Network/Library/Fonts}
          \endrow
        \endtabulate
      \endcentered
    \else
      \setuplocalinterlinespace [14pt]
      \starttabulate [|l|p(.5\textwidth)|]
        \NC Windows   \NC \inlinecode {\% WINDIR\%\\ Fonts} \NC \NR
        \NC Linux     \NC \fileent{/usr/local/etc/fonts/fonts.conf} and\crlf
                          \fileent{/etc/fonts/fonts.conf} \NC \NR
        \NC
          Mac         \NC \fileent{\textasciitilde/Library/Fonts},\crlf
                          \fileent{/Library/Fonts},\break
                          \fileent{/System/Library/Fonts}, and\crlf
                          \fileent{/Network/Library/Fonts} \NC \NR
      \stoptabulate
    \fi%
  }

\endsubsection

\beginsubsection{Querying from Outside}

\fileent{luaotfload-tool} also provides rudimentary means of
accessing the information collected in the font database.
%
If the option \inlinecode {--find=}\emphasis{name} is given, the script will
try and search the fonts indexed by \identifier{luaotfload} for a
matching name.
%
For instance, the invocation

\beginlisting
  luaotfload-tool  --find="Iwona Regular"
\endlisting

\noindent
will verify if “Iwona Regular” is found in the database and can be
readily requested in a document.

If you are unsure about the actual font name, then add the
\inlinecode {-F} (or \inlinecode {--fuzzy}) switch to the command line to enable
approximate matching.
%
Suppose you cannot precisely remember if the variant of
\identifier{Iwona} you are looking for was “Bright” or “Light”.
The query

\beginlisting
  luaotfload-tool  -F --find="Iwona Bright"
\endlisting

\noindent
will tell you that indeed the latter name is correct.

Basic information about fonts in the database can be displayed
using the \inlinecode {-i} option (\inlinecode {--info}).
%
\beginlisting
  luaotfload-tool  -i --find="Iwona Light Italic"
\endlisting
%
\noindent
The meaning of the printed values is described in section 4.4 of the
\LUATEX reference manual.\footnote{%
  In \TEX Live: \fileent{texmf-dist/doc/luatex/base/luatexref-t.pdf}.
}

For a much more detailed report about a given font try the
\inlinecode {-I} option instead (\inlinecode {--inspect}).
\beginlisting
  luaotfload-tool  -I --find="Iwona Light Italic"
\endlisting

\inlinecode {luaotfload-tool --help} will list the available command line
switches, including some not discussed in detail here.
%
For a full documentation of \identifier{luaotfload-tool} and its
capabilities refer to the manpage
(\inlinecode {man 1 luaotfload-tool}).\footnote{%
  Or see \inlinecode {luaotfload-tool.rst} in the source directory.
}

\endsubsection

\beginsubsection {Blacklisting Fonts}
\label{font-blacklist}

Some fonts are problematic in general, or just in \LUATEX.
%
If you find that compiling your document takes far too long or eats
away all your system’s memory, you can track down the culprit by
running \inlinecode {luaotfload-tool -v} to increase verbosity.
%
Take a note of the \emphasis{filename} of the font that database
creation fails with and append it to the file
\fileent{luaotfload-blacklist.cnf}.

A blacklist file is a list of font filenames, one per line.
Specifying the full path to where the file is located is optional, the
plain filename should suffice.
%
File extensions (\fileent{.otf}, \fileent{.ttf}, etc.) may be omitted.
%
Anything after a percent (\inlinecode {\%}) character until the end of the line
is ignored, so use this to add comments.
%
Place this file to some location where the \identifier{kpse}
library can find it, e.~g.
\fileent{texmf-local/tex/luatex/luaotfload} if you are running
\identifier{\TEX Live},\footnote{%
  You may have to run \inlinecode {mktexlsr} if you created a new file in
  your \fileent{texmf} tree.
}
or just leave it in the working directory of your document.
%
\identifier{luaotfload} reads all files named
\fileent{luaotfload-blacklist.cnf} it finds, so the fonts in
\fileent{./luaotfload-blacklist.cnf} extend the global blacklist.

Furthermore, a filename prepended with a dash character (\inlinecode{-}) is
removed from the blacklist, causing it to be temporarily whitelisted
without modifying the global file.
%
An example with explicit paths:

\beginlisting
% example otf-blacklist.cnf
/Library/Fonts/GillSans.ttc  % Luaotfload ignores this font.
-/Library/Fonts/Optima.ttc   % This one is usable again, even if
                             % blacklisted somewhere else.
\endlisting

\endsubsection
\endsection

%%%%%%%%%%%%%%%%%%%%%%%%%%%%%%%%%%%%%%%%%%%%%%%%%%%%%%%%%%%%%%%%%%%%%%%%%%%%%%%
\beginsection {The Fontloader}
%%%%%%%%%%%%%%%%%%%%%%%%%%%%%%%%%%%%%%%%%%%%%%%%%%%%%%%%%%%%%%%%%%%%%%%%%%%%%%%

\beginsubsection {Overview}

To a large extent, \identifier{luaotfload} relies on code originally
written by Hans Hagen for the
\hyperlink[\identifier{\CONTEXT}]{http://wiki.contextgarden.net}
format.
%
It integrates the font loader, written entirely in \LUA, as distributed
in the \identifier{\LUATEX-Fonts} package.
%
The original \LUA source files have been combined using the \CONTEXT
packaging library into a single, self-contained blob. In
this form the font loader depends only on the \identifier{lualibs}
package and requires only minor adaptions to integrate into
\identifier{luaotfload}.

The guiding principle is to let \CONTEXT/\LUATEX-Fonts take care of the
implementation, and update the imported code as frequently as
necessary.
%
As maintainers, we aim at importing files from upstream essentially
\emphasis{unmodified}, except for renaming them to prevent name
clashes.
%
This job has been greatly alleviated since the advent of
\LUATEX-Fonts, prior to which the individual dependencies had to be
manually spotted and extracted from the \CONTEXT source code in a
complicated and error-prone fashion.

\endsubsection

\beginsubsection {Contents and Dependencies}

Below is a commented list of the files distributed with
\identifier{luaotfload} in one way or the other.
%
See see the figure on page \pageref{file-graph} for a
graphical representation of the dependencies.
%
\label{package}%
Through the script \fileent{mkimport} a \CONTEXT library
is invoked to create the \identifier{luaotfload} fontloader as a merged
(amalgamated) source file.\footnote{%
  In \CONTEXT, this facility can be accessed by means of a
  \hyperlink[script]{https://bitbucket.org/phg/context-mirror/src/beta/scripts/context/lua/mtx-package.lua?at=beta}
  which is integrated into \fileent{mtxrun} as a subcommand.
  Run \inlinecode {mtxrun --script package --help} to display further
  information.
  For the actual merging code see the file
  \fileent{util-mrg.lua} that is part of \CONTEXT.
}
%
This file constitutes the “default fontloader” and is part of the
\identifier{luaotfload} package as \fileent{fontloader-YY-MM-DD.lua},
where the uppercase letters are placeholders for the build date.
%
A companion to it, \fileent{luatex-basics-gen.lua} (renamed to \fileent{fontloader-basics-gen.lua} in \identifier{luaotfload})
must be loaded beforehand to set up parts of the environment required by the \CONTEXT
libraries.
%
During a \TEX\ run, the fontloader initialization and injection happens
in the module \fileent{luaotfload-init.lua}.
%
Additionally, the “reference fontloader” as imported from \LUATEX-Fonts
is provided as the file \fileent{fontloader-reference.lua}.
%
This file is self-contained in that it packages all the auxiliary \LUA
libraries too, as Luaotfload did up to the 2.5 series; since that job
has been offloaded to the \identifier{Lualibs} package, loading this
fontloader introduces a certain code duplication.

A number of \emphasis{\LUA utility libraries} are not part of the
\identifier{luaotfload} fontloader, contrary to its equivalent in
\LUATEX-Fonts. These are already provided by the \identifier{lualibs}
and have thus been omitted from the merge.\footnote{%
  Faithful listeners will remember the pre-2.6 era when the fontloader
  used to be integrated as-is which caused all kinds of code
  duplication with the pervasive \identifier{lualibs} package.
  This conceptual glitch has since been amended by tightening the
  coupling with the excellent \CONTEXT\ toolchain.
}

\begindoublecolumns
  \begindefinitions
 \directlua{ printctxlibslist ()}
  \enddefinitions
\enddoublecolumns

The reference fontloader is home to several \LUA files that can be
grouped twofold as below:

\begindefinitions
  \beginnormalitem
    The \emphasis{font loader} itself.
    These files have been written for \LUATEX-Fonts and they are
    distributed along with \identifier{luaotfload} so as to resemble
    the state of the code when it was imported. Their purpose is either
    to give a slightly aged version of a file if upstream considers
    latest developments for not yet ready for use outside Context; or,
    to install placeholders or minimalist versions of APIs relied upon
    but usually provided by parts of Context not included in the
    fontloader.
    \begindoublecolumns
      \begindefinitions
        \directlua{printctxallgenericlist ()}
      \enddefinitions
    \enddoublecolumns
  \endnormalitem

  \beginnormalitem
    Code related to \emphasis{font handling and node processing}, taken
    directly from \CONTEXT.
    \begindoublecolumns
      \begindefinitions
      \directlua{printctxfontlist ()}
      \enddefinitions
    \enddoublecolumns
  \endnormalitem
\enddefinitions

As an alternative to the merged file, \identifier {Luaotfload} may load
individual unpackaged \LUA libraries that come with the source, or even
use the files from Context directly.
%
Thus if you prefer running bleeding edge code from the \CONTEXT beta,
choose the \inlinecode {context} fontloader via the configuration file
(see sections \ref{sec:conf} and \ref{sec:pkg} below).

Also, the merged file at some point loads the Adobe Glyph List from a
\LUA table that is contained in \fileent{luaotfload-glyphlist.lua},
which is automatically generated by the script
\fileent{mkglyphlist}.\footnote{%
  See \fileent{luaotfload-font-enc.lua}.
  The hard-coded file name is why we have to replace the procedure
  that loads the file in \fileent{luaotfload-init.lua}.
}
%
There is a make target \identifier{glyphs} that will create a fresh
glyph list so we don’t need to import it from \CONTEXT any longer.

In addition to these, \identifier{luaotfload} requires a number of
files not contained in the merge. Some of these have no equivalent in
\LUATEX-Fonts or \CONTEXT, some were taken unmodified from the latter.


\beginfilelist
    \beginaltitem {luaotfload-features.lua}
      font feature handling; incorporates some of the code from
      \fileent{font-otc} from \CONTEXT;
    \endaltitem
    \beginaltitem {luaotfload-configuration.lua}
      handling of \fileent{luaotfload.conf(5)}.
    \endaltitem
    \beginaltitem {luaotfload-log.lua}
      overrides the \CONTEXT logging functionality.
    \endaltitem
    \beginaltitem {luaotfload-loaders.lua}
      registers readers in the fontloader for various kinds of
      font formats
    \endaltitem
    \beginaltitem {luaotfload-parsers.lua}
      various \abbrev{lpeg}-based parsers.
    \endaltitem
    \beginaltitem {luaotfload-database.lua}
      font names database.
    \endaltitem
    \beginaltitem {luaotfload-resolvers.lua}
      file name resolvers.
    \endaltitem
    \beginaltitem {luaotfload-colors.lua}
      color handling.
    \endaltitem
    \beginaltitem {luaotfload-auxiliary.lua}
      access to internal functionality for package authors (proposals
      for additions welcome).
    \endaltitem
    \beginaltitem {luaotfload-letterspace.lua}
      font-based letterspacing.
    \endaltitem
        \beginaltitem {luaotfload-filelist.lua}
      data about the files in the package.
    \endaltitem
\endfilelist

%\figurefloat
%  {file-graph}
%  {Schematic of the files in \identifier{Luaotfload}}
%  {filegraph.pdf}

\endsubsection

\beginsubsection {Packaging}

\label{sec:pkg}%
The fontloader code is integrated as an isolated component that can be
switched out on demand.
%
To specify the fontloader you wish to use, the configuration file
(described in section \ref{sec:conf}) provides the option
\inlinecode{fontloader}.
%
Its value can be one of the identifiers \inlinecode{default} or
\inlinecode{reference} (see above, section \ref{package}) or the name
of a file somewhere in the search path of \LUATEX.
%
This will make \identifier {Luaotfload} locate the \CONTEXT source by
means of \identifier{kpathsea} lookups and use those instead of the
merged package.
%
The parameter may be extended with a path to the \CONTEXT
\fileent{texmf}, separated with a colon:

\beginlisting
[run]
  fontloader = context:~/context/tex/texmf-context
\endlisting

\noindent This setting allows accessing an installation -- e. g. the
standalone distribution or a source repository -- outside the current
\TEX distribution.

Like the \identifier{Lualibs} package, the fontloader is deployed as a
\emphasis{merged package} containing a series of \LUA files joined
together in their expected order and stripped of non-significant parts.
%
The \fileent{mkimport} utility assists in pulling the files from a
\CONTEXT tree and packaging them for use with \identifier{Luaotfload}.%
%
The state of the files currently in \identifier{Luaotfload}’s
repository can be queried:
\beginlisting
./scripts/mkimport news
\endlisting
%
The subcommand for importing takes the prefix of the desired \CONTEXT
\identifier{texmf} as an optional argument:
\beginlisting
./scripts/mkimport import ~/context/tex/texmf-context
\endlisting
%
Whereas the command for packaging requires a path to the
\emphasis{package description file} and the output name to be passed.
\beginlisting
./scripts/mkimport package fontloader-custom.lua
\endlisting

From the toplevel makefile, the targets \inlinecode{import} and
\inlinecode{package} provide easy access to the commands as invoked during
the \identifier{Luaotfload} build process.\footnote{%
  \emphasis{Hint for those interested in the packaging process}: issue
  \inlinecode{make show} for a list of available build routines.
}
These will call \inlinecode{mkimport} script with the correct
parameters to generate a datestamped package.
%
Whether files have been updated in the upstream distribution can be
queried by \inlinecode{./scripts/mkimport news}.
%
This will compare the imported files with their counterparts in the
\CONTEXT distribution and report changes.

\endsubsection

\endsection

%%%%%%%%%%%%%%%%%%%%%%%%%%%%%%%%%%%%%%%%%%%%%%%%%%%%%%%%%%%%%%%%%%%%%%%%%%%%%%%
\beginsection {Configuration Files}
%%%%%%%%%%%%%%%%%%%%%%%%%%%%%%%%%%%%%%%%%%%%%%%%%%%%%%%%%%%%%%%%%%%%%%%%%%%%%%%

\beginnarrower
  \emphasis{Caution}: For the authoritative documentation, consult the
  manpage for \fileent{luaotfload.conf(5)}.
\endnarrower

\label{sec:conf}
The runtime behavior of \identifier{Luaotfload} can be customized by
means of a configuration file.
% location
At startup, it attempts to locate a file called \fileent
{luaotfload.conf} or \fileent {luaotfloadrc} at a number of candidate
locations:

\begincentered
  \begindefinitions
    \beginnormalitem \fileent{./luaotfload.conf}                            \endnormalitem
    \beginnormalitem \fileent{./luaotfloadrc}                               \endnormalitem
    \beginnormalitem \fileent{\$XDG_CONFIG_HOME/luaotfload/luaotfload.conf} \endnormalitem
    \beginnormalitem \fileent{\$XDG_CONFIG_HOME/luaotfload/luaotfload.rc}   \endnormalitem
    \beginnormalitem \fileent{~/.luaotfloadrc}                              \endnormalitem
  \enddefinitions
\endcentered

\beginnarrower
  \emphasis{Caution}: The configuration potentially modifies the final
  document. A project-local file belongs under version control along
  with the rest of the document. This is to ensure that everybody who
  builds the project also receives the same customizations as the
  author.
\endnarrower

% syntax
The syntax is fairly close to the format used by
\fileent{git-config(1)} which in turn was derived from the popular
\identifier{.INI} format: Lines of key-value pairs are grouped under
different configuration “sections”.\footnote{%
  The configuration parser in \fileent {luoatfload-parsers.lua} might
  be employed by other packages for similar purposes.
}
% example settings
An example for customization via \fileent {luaotfload.conf} might look
as below:

\beginlisting
; Example luaotfload.conf containing a rudimentary configuration
[db]
  update-live = false
[run]
  color-callback = pre_linebreak_filter
  definer = info_patch
  log-level = 5
[default-features]
  global = mode=base
\endlisting

This specifies that for the given project, \identifier{Luaotfload}
shall not attempt to automatically scan for fonts if it can’t resolve a
request. The font-based colorization will happen during \LUATEX’s
pre-linebreak filter. The fontloader will output verbose information
about the fonts at definition time along with globally increased
verbosity. Lastly, the fontloader defaults to the less expensive
\luaident{base} mode like it does in \CONTEXT.

%%%%%%%%%%%%%%%%%%%%%%%%%%%%%%%%%%%%%%%%%%%%%%%%%%%%%%%%%%%%%%%%%%%%%%%%%%%%%%%
\beginsection {Auxiliary Functions}
%%%%%%%%%%%%%%%%%%%%%%%%%%%%%%%%%%%%%%%%%%%%%%%%%%%%%%%%%%%%%%%%%%%%%%%%%%%%%%%

With release version 2.2, \identifier{Luaotfload} received
additional functions for package authors to call from outside
(see the file \fileent{luaotfload-auxiliary.lua} for details).
%
The purpose of this addition twofold.
%
Firstly, \identifier{luaotfload} failed to provide a stable interface
to internals in the past which resulted in an unmanageable situation
of different packages abusing the raw access to font objects by means
of the \luaident{patch_font} callback.
%
When the structure of the font object changed due to an update, all
of these imploded and several packages had to be fixed while
simultaneously providing fallbacks for earlier versions.
%
Now the patching is done on the \identifier{luaotfload} side and can
be adapted with future modifications to font objects without touching
the packages that depend on it.
%
Second, some the capabilities of the font loader and the names
database are not immediately relevant in \identifier{luaotfload}
itself but might nevertheless be of great value to package authors or
end users.

Note that the current interface is not yet set in stone and the
development team is open to suggestions for improvements or
additions.

\beginsubsection {Callback Functions}

The \luaident{patch_font} callback is inserted in the wrapper
\identifier{luaotfload} provides for the font definition callback.
%
At this place it allows manipulating the font object immediately after
the font loader is done creating it.
%
For a short demonstration of its usefulness, here is a snippet that
writes an entire font object to the file \fileent{fontdump.lua}:

\beginlisting
  \input luaotfload.sty
  \directlua{
    local dumpfile    = "fontdump.lua"
    local dump_font   = function (tfmdata)
      local data = table.serialize(tfmdata)
      io.savedata(dumpfile, data)
    end

    luatexbase.add_to_callback(
      "luaotfload.patch_font",
      dump_font,
      "my_private_callbacks.dump_font"
    )
  }
  \font \dumpme = name:Iwona
  \bye
\endlisting

\emphasis{Beware}: this creates a Lua file of around 150,000 lines of
code, taking up 3~\abbrev{mb} of disk space.
%
By inspecting the output you can get a first impression of how a font
is structured in \LUATEX’s memory, what elements it is composed of,
and in what ways it can be rearranged.

\beginsubsubsection {Compatibility with Earlier Versions}

As has been touched on in the preface to this section, the structure
of the object as returned by the fontloader underwent rather drastic
changes during different stages of its development, and not all
packages that made use of font patching have kept up with every one
of it.
%
To ensure compatibility with these as well as older versions of
some packages, \identifier{luaotfload} sets up copies of or references
to data in the font table where it used to be located.
%
For instance, important parameters like the requested point size, the
units factor, and the font name have again been made accessible from
the toplevel of the table even though they were migrated to different
subtables in the meantime.

\endsubsubsection

\beginsubsubsection{Patches}

These are mostly concerned with establishing compatibility with \XETEX.

\beginfunctionlist

  \beginaltitem  {set_sscale_dimens}
    Calculate \texmacro{fontdimen}s 10 and 11 to emulate \XETEX.
  \endaltitem

  \beginaltitem  {set_capheight}
    Calculates \texmacro{fontdimen} 8 like \XETEX.
  \endaltitem

  \beginaltitem  {patch_cambria_domh}
    Correct some values of the font \emphasis{Cambria Math}.
  \endaltitem

\endfunctionlist

\endsubsection

\beginsubsection {Package Author’s Interface}

As \LUATEX release 1.0 is nearing, the demand for a reliable interface
for package authors increases.

\endsubsubsection

\beginsubsubsection{Font Properties}

Below functions mostly concern querying the different components of a
font like for instance the glyphs it contains, or what font features
are defined for which scripts.

\beginfunctionlist

  \beginaltitem  {aux.font_has_glyph (id : int, index : int)}
            Predicate that returns true if the font \luaident{id}
            has glyph \luaident{index}.
  \endaltitem

  \beginaltitem  {aux.slot_of_name(id : int, name : string)}
            Translates a name for a glyph in font \luaident{id} to the
            corresponding glyph slot which can be used e.g.\ as an argument to
            \inlinecode{\char}.
  \endaltitem

  \beginaltitem  {aux.name_of_slot(id : int, slot : int)}
            The inverse of \luaident{slot_of_name}; note that this
            might be incomplete as multiple glyph names may map to the
            same codepoint, only one of which is returned by
            \luaident{name_of_slot}.
  \endaltitem

  \beginaltitem  {aux.gid_of_name(id : int, name : string)}
            Translates a Glyph name to the corresponding GID in font
            \luaident{id}. This corresponds to the value returned by
            \inlinecode{\XeTeXglyphindex} in \XeTeX.
  \endaltitem

  \beginaltitem  {aux.provides_script(id : int, script : string)}
            Test if a font supports \luaident{script}.
  \endaltitem

  \beginaltitem  {aux.provides_language(id : int, script : string, language : string)}
            Test if a font defines \luaident{language} for a given
            \luaident{script}.
  \endaltitem

  \beginaltitem  {aux.provides_feature(id : int, script : string,
             language : string, feature : string)}
            Test if a font defines \luaident{feature} for
            \luaident{language} for a given \luaident{script}.
  \endaltitem

  \beginaltitem  {aux.get_math_dimension(id : int, dimension : string)}
            Get the dimension \luaident{dimension} of font \luaident{id}.
  \endaltitem

  \beginaltitem  {aux.sprint_math_dimension(id : int, dimension : string)}
            Same as \luaident{get_math_dimension()}, but output the value
            in scaled points at the \TEX end.
  \endaltitem

\endfunctionlist

\endsubsubsection

\beginsubsubsection{Database}

%% not implemented, may come back later
\beginfunctionlist
%   \beginaltitem  {aux.scan_external_dir(dir : string)}
%             Include fonts in directory \luaident{dir} in font lookups without
%             adding them to the database.
%
  \beginaltitem  {aux.read_font_index (void)}
            Read the index file from the appropriate location (usually
            the bytecode file \fileent{luaotfload-names.luc} somewhere
            in the \fileent{texmf-var} tree) and return the result as a
            table. The file is processed with each call so it is up to
            the user to store the result for later access.
  \endaltitem

  \beginaltitem  {aux.font_index (void)}
            Return a reference of the font names table used internally
            by \identifier{luaotfload}. The index will be read if it
            has not been loaded up to this point. Also a font scan that
            overwrites the current index file might be triggered. Since
            the return value points to the actual index, any
            modifications to the table might influence runtime behavior
            of \identifier{luaotfload}.
  \endaltitem

\endfunctionlist

\endsubsubsection

\endsubsection
\endsection

%%%%%%%%%%%%%%%%%%%%%%%%%%%%%%%%%%%%%%%%%%%%%%%%%%%%%%%%%%%%%%%%%%%%%%%%%%%%%%%
\beginsection {Troubleshooting}
%%%%%%%%%%%%%%%%%%%%%%%%%%%%%%%%%%%%%%%%%%%%%%%%%%%%%%%%%%%%%%%%%%%%%%%%%%%%%%%

\beginsubsection {Database Generation}

If you encounter problems with some fonts, please first update to the
latest version of this package before reporting a bug, as
\identifier{luaotfload} is under active development and still a moving
target.
%
The development takes place on \identifier{github} at
\hyperlink {https://github.com/lualatex/luaotfload} where there is an issue
tracker for submitting bug reports, feature requests and the likes.

Bug reports are more likely to be addressed if they contain the output
of

\beginlisting
    luaotfload-tool --diagnose=environment,files,permissions
\endlisting

\noindent Consult the man page for a description of these options.

Errors during database generation can be traced by increasing the
verbosity level and redirecting log output to \fileent{stdout}:

\beginlisting
    luaotfload-tool -fuvvv --log=stdout
\endlisting

\noindent or to a file in \fileent{/tmp}:

\beginlisting
    luaotfload-tool -fuvvv --log=file
\endlisting

\noindent In the latter case, invoke the \inlinecode {tail(1)} utility on the
file for live monitoring of the progress.

If database generation fails, the font last printed to the terminal or
log file is likely to be the culprit.
%
Please specify it when reporting a bug, and blacklist it for the time
being (see above, page \pageref{font-blacklist}).

\endsubsection

\beginsubsection {Font Features}

A common problem is the lack of features for some
\OpenType fonts even when specified.
%
This can be related to the fact that some fonts do not provide features
for the \inlinecode {dflt} script (see above on page \pageref{script-tag}),
which is the default one in this package.
%
If this happens, assigning a noth script when the font is defined should
fix it.
%
For example with \inlinecode {latn}:

\beginlisting
    \font \test = file:MyFont.otf:script=latn;+liga;
\endlisting

You can get a list of features that a font defines for scripts and
languages by querying it in \fileent{luaotfload-tool}:

\beginlisting
    luaotfload-tool --find="Iwona" --inspect
\endlisting

\endsubsection

\beginsubsection {\LUATEX Programming}

Another strategy that helps avoiding problems is to not access raw
\LUATEX internals directly.
%
Some of them, even though they are dangerous to access, have not been
overridden or disabled.
%
Thus, whenever possible prefer the functions in the \luaident{aux}
namespace over direct manipulation of font objects. For example, raw
access to the \luaident{font.fonts} table like:

\beginlisting
    local somefont = font.fonts[2]
\endlisting

\noindent can render already defined fonts unusable.
%
Instead, the function \luaident{font.getfont()} should be used
because it has been replaced by a safe variant.

However, \luaident{font.getfont()} only covers fonts handled by the
font loader, e.~g. \identifier{OpenType} and \identifier{TrueType}
fonts, but not \abbrev{tfm} or \abbrev{ofm}.
%
Should you absolutely require access to all fonts known to \LUATEX,
including the virtual and autogenerated ones, then you need to query
both \luaident{font.getfont()} and \luaident{font.fonts}.
%
In this case, best define you own accessor:

\beginlisting
    local unsafe_getfont = function (id)
        local tfmdata = font.getfont (id)
        if not tfmdata then
            tfmdata = font.fonts[id]
        end
        return tfmdata
    end

    --- use like getfont()
    local somefont = unsafe_getfont (2)
\endlisting

\endsubsection
\endsection

\beginsection {License}

\identifier {luaotfload} is licensed under the terms of the
\hyperlink [GNU General Public License version 2.0]%
           {https://www.gnu.org/licenses/old-licenses/gpl-2.0.html}.
Following the underlying fontloader code \identifier {luaotfload}
recognizes only that exact version as its license.
The „any later version” clause of the original license text as
copyrighted by the \hyperlink [Free Software Foundation]{http://www.fsf.org/}
\emphasis {does not apply} to either \identifier {luaotfload} or the
code imported from \CONTEXT.

The complete text of the license is given as a separate file \fileent
{COPYING} in the toplevel directory of the
\hyperlink [\fileent {Luaotfload} Git repository]{https://github.com/lualatex/luaotfload/blob/master/COPYING}.\\
Distributions probably package it as \fileent
{doc/luatex/luaotfload/COPYING} in the relevant \fileent {texmf} tree.

\endsection

\endinput

% vim:ft=tex:tw=79:et:sw=2
